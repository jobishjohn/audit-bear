\section{Related Work}
Many election technology systems provide a means of auditing elections. For example, in optical scanning systems the cast ballots form a paper record of the votes cast.  On the other hand, paperless DRE machines do not provide this type of paper trail. Some DRE systems provide a Voter Verified Paper Audit Trail (VVPAT), which stores a hard copy version of each ballot cast.  A third type of audit trail, which is produced by all DREs, are the event logs stored electronically on each DRE.  In this section we discuss related work on the analysis of audit logs for post-election auditing. 

Two recent studies used event logs from the iVotronic voting system to audit elections~\cite{Buell2011,Sandler2007}. The authors of the first study~\cite{Buell2011} performed an audit of the same South Carolina elections that we analyzed. Using these audit logs, they discovered votes not included in the certified counts and problems with the audit data. They crosschecked the event log, the ballot image file, and the system log to identify unsupported votes and missing audit data.  By consulting additional audit materials, such as the printed results tapes, the authors were able to offer possible reasons and explanations as to why the problems occurred. Our work takes a slightly different approach.  We focus on developing a variety of methods to analyze the publicly available audit log data; in addition, we automate these analyses for use by election officials and interested third parties.  While our tool did discover and report similar problems, we simply report what was wrong, but can not provide a possible explanation for the cause of the error as we didn't have access to printed results tapes. 

The authors of the second study~\cite{Sandler2007} provided an analysis of vote tallies using the protected count of votes on each machine and comparing this to the printed results tapes. Their report also finds date/time stamps that were most likely inaccurate.  With further investigation, they concluded that the machine hardware clock was incorrect. Our research provides analyses to identify similar problems, but in a way that could be automated. 

There has also been research on using the audit logs to analyze election-day procedure and activity. For example, one recent publication showed how event logs could be used to determine if a machine acted \textquotedblleft normally\textquotedblright on election day~\cite{Antonyan2009}. The authors of this research studied the event logs of the AccuVote Optical Scanning system and used those logs to build a finite state machine that models the sequences of events that a well-behaved machine might produce. This type of analysis would be useful to provide for the iVotronic systems that we studied. However, the AccuVote machines have considerably fewer possible event types than the iVotronics so the analysis would become considerably more complex. 

A common problem on election day, which we try to identify in our analysis, is the occurrence of long lines. Many studies have researched ways to mitigate long lines at polling locations ~\cite{Allen2006,Dow2007,Spencer2010,Wilson2008,Edel2010}.  One such study has simulated the flow of voters through the voting process~\cite{Edel2010}. The authors use this simulation to determine the optimal number of voters per voting machine, and correspondingly, the appropriate number of voting machines per polling location based on the number of registered voters at that particular location. Their work is predictive: the authors make some assumptions, such as the average time it takes to vote and when peak voting hours will occur, and use those as a basis for predicting where long lines are likely to occur. Our analysis is descriptive: given the audit logs from election day, we infer the average time it took to vote and use that information to determine whether a particular polling location experienced long lines or not. The two methods are complementary. Predictive models can be used to prevent long lines, while descriptive models can be used to check and refine the prediction algorithms. 

Voter Verified Paper Audit Trails (VVPATs) are a different type of audit log. Unlike the audit logs we used in our analyses, VVPATs are viewed and verified by the voter and are more suited to audits concerning a DRE incorrectly capturing a voter\textquoteright s intent. Our work is more concerned with identifying cases of cast votes not being included in the final count, or issues at the polling place that might prevent the voter from casting their vote in the first place. With VVPATs, as long as a certain percentage of voters do check their paper ballot~\cite{Hall2006}, the voting machine need not be assumed correct, whereas our analyses do make this assumption.

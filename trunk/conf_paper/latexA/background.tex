\section{Background}
\label{sec:background}
\subsection{Introduction to the iVotronic DRE}
A brief description of the iVotronic's functionality and its main
system components follows. 

\begin{itemize} 
\item \textbf{Voting terminal.} The voting terminal is a stand-alone
  touchscreen voting unit.  The terminal is equipped with an internal
  battery, which keeps the unit operational in the event of a power
  failure, and a removable compact flash card, which is used to store
  audit data and ballot images (cast vote records). Typically, each
  polling location is assigned several iVotronic machines. Also, to
  comply with the federal American with Disabilities Act, at least one
  audio terminal is placed in each precinct to assist disabled
  voters. 

\item \textbf{Personalized Electronic Ballot (PEB).} The PEB is a
  proprietary cartridge designed by ES\&S to operate the iVotronic
  terminal.  When the PEB is placed in the machine, the terminal and
  the PEB can communicate through an infrared port.  Typically,
  counties deploy two types of PEBs to the precinct: a) a master PEB
  and b) an activator PEB. Both types of PEBs have the same
  functionality, however, poll workers are trained to keep them
  separate and use them for different purposes. 
    \begin{itemize}
    \item \textbf{Master PEB.}  Poll workers use the master PEB to
      open polls on election day. When the PEB is placed in the
      terminal, the touchscreen displays the precinct's name
      programmed in the PEB so that poll workers can verify the
      polling location information and date/time registered in the
      terminal's internal clock. If the information displayed is
      correct, the poll workers open the terminal for voting. The same
      master PEB should be used to open all terminals in the polling
      location. In the same fashion, the master PEB should be used to
      close all terminals in the polling location at the end of the
      voting day. When a terminal is closed, it uploads its vote
      totals onto the PEB inserted into it. When the master PEB is
      used to close all terminals, this PEB accumulates the precinct
      totals so that they can later be uploaded and included in the
      official tally. 
    \item \textbf{Activator PEBs.}  Activator PEBs are used by  poll
      workers to activate ballots for voters. Election officials
      provide each precinct with multiple activator PEBs.  
    \end{itemize}
\end{itemize}
Internally, all PEBs at the precinct are identical. The only
difference between them is the color of the rubber band on their
exterior. Thus, a master PEB can be used to activate a voter's ballot
and an activator PEB can be used to open and close terminals. Though
as a matter of procedure and training, they should not be used this
way. If an activator PEB is used to close terminals, the precinct vote
totals may be only partially uploaded to the aggregated totals on
election night. Poll workers are trained so that they put the master
PEB, CF cards and precinct's totals tapes in a designated bag that is
transported to Election Central after polls close.  Activator PEBs
used to close terminals may be left behind and their vote data not
added to the certified count.
\begin{itemize}
\item \textbf{Removable Compact Flash (CF) card.} The CF cards are
  programmed at Election Central and installed in the back of the
  voting terminal prior to deployment at the polling location. The CF
  cards contain graphic (bitmap) files read by the voting terminal
  during the voting process. The CF cards are also used as an external
  memory device: audit log entries and ballot images are written to
  the CF card when the terminal is closed for voting. Once the polls
  close, the CF cards are removed from the back of the terminal and
  delivered to election headquarters on election night.  

\item \textbf{External printer module.} This module is connected to a
  serial port on the back of the voting terminal. The thermal printer
  produces the precinct zero tape and results tape. Poll workers are
  instructed to print the zero tape once all iVotronics of the polling
  location are opened for voting. In the same fashion, they are
  trained to print the results tape when all voting terminals are
  closed on election night. 
\end{itemize}

\subsection{iVotronic Audit}
The ES\&S voting solution produces many log files including the
Election Reporting Manager (ERM) \textquotedblleft System
Log\textquotedblright \hspace{1 mm} file, the ERM \textquotedblleft
Result Correction Log,\textquotedblright \hspace{1 mm} the ERM
\textquotedblleft Real Time Log,\textquotedblright \hspace{1 mm} the
iVotronic \textquotedblleft Ballot Image\textquotedblright \hspace{1
  mm} log and the iVotronic \textquotedblleft Event
Log.\textquotedblright \hspace{2 mm} We focus on three: the event log
(EL152.lst), ballot image file (EL155.lst), and the system log
(EL68a.lst).  The header of these log files identify the county's
name, the type and date of the election, the date the report was
generated, and the election ID. The election ID is a parameter
generated by the ES\&S election programming software to uniquely
identify each election.  
 
The event log (EL152.lst) contains audit log entries from each
iVotronic terminal used in the election.  The log  records, in
chronological order, all events that occurred on that machine in the
election. It typically begins at election headquarters, before the
election, with a \textquotedblleft clear and
test\textquotedblright \hspace{1 mm} of the terminal to delete
previous election data from the terminal's memory. It also records all
election day events, including polls open and polls closing and the
number of ballots cast.  Each event log entry includes the iVotronic's
terminal serial number, the PEB's serial number, the date and time,
the event that occurred and a description of the event. An excerpt of
an event log is given in  Appendix~\ref{app:el}. 
 
The ballot image file (EL155.lst) contains all ballot images saved by
the iVotronic terminals during the voting process. An ES\&S ballot
image is a list of all choices made by a single voter (i.e., a cast
vote record); it is not a scanned or photographic image. The ballot
images are segregated by precinct and terminal where the votes were
cast. The ballots are saved in a random order to protect the privacy
of the voter.  An excerpt of a ballot image file is given in
Appendix~\ref{app:bi}
 
The system log listing file (EL68a.lst) tracks activity in the
election reporting database since its creation at the election
headquarters. Its chronological entries reflect the commands executed
by the operators during  pre-election testing, election night
reporting, post-election testing and canvassing. It also contains the
totals accumulated in the various precincts during election night
reporting, as well as any warnings or errors reported by the reporting
software during the tabulation process. The system log also tracks the
uploading of PEBs and CF cards to the election reporting
database. Manual adjustment of precinct totals are also recorded in
the system log file. An excerpt of a system log file is given in
Appendix~\ref{app:sl}. 

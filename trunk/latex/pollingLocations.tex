\subsection{Polling Location Related Analyses}
\subsubsection{Polling Locations That Closed Late}
In the United Stated, poll hours are regulated by the state officials. Poll closing hours vary from 6pm to 9pm depending on the state [footnote: http://www.infoplease.com/us/government/voter-registration-deadlines-poll-hours.html] Some state statutes allow the voters waiting in line at poll closing time to cast their ballots at the precinct; therefore,  polling locations may stay open late in order to accommodate those voters prior to closing the voting machines on election night. If election officials knew which polling places were likely to experience long lines they could deploy more equipment or personnel to those locations. Our analysis can assist them by providing information about long lines that occurred in this election. Election officials can use this information to make predictions about where long lines might occur in future elections.

This analysis gathers the time each machine was closed from the event log and the precinct the machine was assigned to from the ballot image file; to perform this analysis, this data is required.  Our algorithm saves this information in a data structure and generates a countywide chart detailing the number of polling locations that stayed open after poll closing time grouped by 10 minute increments. To handle inaccurate machine time/date settings, our tool uses a time verification algorithm described in section 3.6 to exclude from our database any voting machines whose time stamp is probably incorrect. This analysis gives election officials information about how many polling locations had to stay open late and for how long.

\subsubsection{Long Lines}
Election officials assign voting machines and polling location supplies based on the number of voters registered in each precinct.  However, voter turnout can vary and as a result, some polling locations may end up overstocked with equipment, supplies or poll workers while others may lack resources or personnel on election day. Monitoring all the polling places in a large county can be a daunting task. Often, election officials don't have any process in place to monitor polling location congestion. More often than not, DRE jurisdictions experience bottlenecks caused by voters waiting inside the precinct for a voting machine to cast their ballot. This is common during mid-term and general elections~\cite{Kreitman2010, Slade2008, U2010}.  In those circumstances, counties can benefit from a tool that can analyze DRE audit data to identify peak times at the precincts. Such information can assist election officials with the planning of future elections by augmenting voting machines or resources where they may be lacking.

This analysis focuses on lines of voters by detecting heavily used voting terminals. It reports the possibility of busy timeframes due to the number of voting machines. We can infer a steady flow of voters from two logs files: the event log for vote cast time stamp information, and the ballot image file for precinct location information. Accurate vote time stamps are significant in determining which machines were heavily used at the precinct in a specific time period. When there are consecutive ballots cast in all machines in a polling location with no time delay in between, we are able to infer that there is a line of voters waiting to use the voting machines. 

To infer long line, we focused on the polling locations which stayed opened after poll closing time as we could conclude they were busy processing the voters standing in line at that time. For those locations, we calculate the time it took voters to cast ballots during the extended poll hours. We also calculate and keep track of the time per ballot cast during regular poll hours. Using the Kolmogorov- Smirnov statistical test we can determine whether the votes cast during regular poll hours, in one-hour time windows, match the vote pattern distribution calculated during the extended poll hours. If the distribution of the two samples is consistent, we can infer the possibility of long lines during the particular one-hour time period compared with the extended hours vote distribution.

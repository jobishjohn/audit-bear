\subsection{Polling Location Related Analyses}
\subsubsection{Polling Locations that Closed Late}
In the United Stated, poll hours are regulated by the state
officials. Poll closing hours vary from 6pm to 9pm depending on the
state~\cite{Info2007}. Some state statutes allow the voters waiting in
line at poll closing time to cast their ballots at the precinct;
therefore, polling locations may stay open late in order to
accommodate those voters waiting in line at the official poll closing
time. If election officials knew which polling places were likely to
experience long lines they could deploy more equipment or personnel to
those locations. This analysis can assist them by providing
information about long lines that occurred in the current election. Election
officials can use this information to make predictions about where
long lines might occur in future elections of the same type. 

This analysis gathers the time each machine was closed from the event log and the precinct the machine was assigned to from the ballot image file; to perform the analysis, this data is required.  Our algorithm saves this information in a data structure and generates a countywide chart detailing the number of polling locations that stayed open after poll closing time and for how long. To handle inaccurate machine date/time settings, our tool uses a time verification algorithm described in Section 3.6 to exclude from our database any voting machines whose time stamp is probably incorrect. 

\subsubsection{Long Lines}
Election officials assign voting machines and supplies to each polling
location based on the number of voters registered in the precinct.
However, voter turnout can vary and as a result, some polling
locations may end up overstocked with equipment, supplies or poll
workers while others may lack resources on election day. As a result,
voters may have to wait in line before casting their vote. This is
common during mid-term and general elections~\cite{Kreitman2010,
  Slade2008, U2010}, as well as any elections with high-voter turnout.
In those circumstances, election officials might like information
about which locations experienced long lines and at what time of the
day. However, monitoring all the polling places in a large county can
be a daunting task. Therefore, we provide an analysis that can analyze
DRE audit data to identify peak times at the precincts. Such
information can assist election officials with the planning of
resource allocation for future elections.

This analysis reports busy locations by detecting heavily used voting terminals. It requires an event log that captures every vote cast event along with an accurate time stamp and precinct location information. In our analysis we find this information in the event log and the ballot image file. When there are consecutive ballots cast in all machines in a polling location with no time delay in between, we are able to infer that there is a steady flow of voters and possibly a line of voters waiting to use the voting machines. 

However, the analysis is not quite so straightforward. The event logs we used captured the \textquotedblleft cast vote'' event, but that only tells us when the voter finished making their selection. In order to know if there is any idle time between voters, we would also need an event that notes when a new ballot is loaded into a machine. The idle time would be the time between one \textquotedblleft cast vote'' event and the next \textquotedblleft ballot loaded'' event. Unfortunately our logs have no such \textquotedblleft ballot loaded'' event. Instead, we had to infer the idle time. We did this by focusing on the polling locations which stayed opened after poll closing time as we could conclude they were busy processing the voters standing in line at that time and the time between cast vote events includes no idle time. For those locations, we measure the time it took voters to cast ballots during the extended poll hours. We also calculate and keep track of the time per ballot cast during regular poll hours. Using the Kolmogorov- Smirnov statistical test we can determine whether the distribution of time between vote cast events during regular poll hours, in one-hour time windows, matches the distribution of time between vote cast events during the extended poll hours. If the distribution of the two samples is consistent, we can infer the possibility of long lines during the particular one-hour time period. \footnote{The KS test starts with the null hypothesis that the two samples come from the same distribution, which is the situation we are interested in identifying. Therefore, the most we can say is the case that the null hypothesis cannot be rejected, we cannot reject the possibility that there were long lines between a particular time window.}

\subsection{Systematic Date and Time Errors}\label{an:date} 

Correct time stamps are critical in post election audits as they can preclude further log analysis.  Voting machines' audit logs may experience incorrect date and time stamps due to a number of reasons.  We found two distinct categories of problems: errors resulting from incorrectly set clocks and errors resulting from apparent bugs in the time-stamp mechanism itself.  The first type of error indicates that the opening event and possibly all vote events were recorded with incorrect time stamps.  This can prevent other analyses from properly detecting how long the polls stayed open, whether they opened on time, or experienced long lines.  It may be useful for an election official to identify counties or precincts where these errors were particularly common so that pre-election testing procedures can be improved.  The second type of error represents anomalies in the time stamps that aren't a result of human error.  These occurrences suggest an issue with the specific machine firmware or the software itself.  Reporting these anomalies may assist those officials in determining which machines have consistent date issues or in reporting the issue to the DRE vendor.  

To detect both types of date and time stamp errors, the logs produced by the voting machine must record the date and time of each event as well as specific events.  These events include the opening and closing of machines and manual date/time changes.  It is also essential that the logs show recordings in chronological order.  For the benefit of the election officials, the ballot image file will provide data to match machines to their polling locations.  All data was collected by sequentially parsing each time-stamp while keeping track of state. State was determined by looking at events such as the terminal being opened for voting. Only the last set of openings and closings were considered for each machine. This helped ensure we were looking at election day statistics.  This algorithm detects three different situations: machines that had their clocks adjusted during election day voting, machines that opened on incorrect dates for voting, but never had their clocks adjusted correctly, and machines that had anomalies among the time stamps.  Our analysis runs through the event log and simultaneously finds these errors.  For the first situation, the analysis finds machines that were off only by a couple of hours.  We only know this because an official changed the clock during election day.  The algorithm looked for a manual time change event for any machine that was in an open state during election day.  If the machine then closed successfully on election day, the delta of the time change is recorded and factored into the opening time for other analyses.  Another way we detected this situation was by looking for machines that open on an improbable date, but still close on election day.  This included opening on days past election day or days far enough in the past so that early voting machines were not recorded.  For the second scenario, our method finds machines that open and close on improbable dates.  Improbable dates is defined as being a month before election day or being 1 or more days after election day.  By parsing through the chronologically ordered date/time stamps, we can detect these machines.  To track the last situation, we look for any time stamp that decreases chronologically without a manual date adjustment event as well as significant forward date/time jumps.  Forward jumps in time are not necessarily the result of an inconsistent time record.  A machine can be shut down for weeks between events.  Given the relatively short duration of elections, jumps forward in time that far past a certain threshold can reasonably be considered wrong.  This analysis looked for any deltas between time stamps that jump forward by more than 33 days.  This may seem arbitrary, but upon inspection of a specific data set, this seemed to avoid ambiguous jumps.  Additionally, some machines were found to have their clock set to some type of null state where the clock stopped incrementing and all of the time stamps were marked as a zero date.  This last situation will report the number of events associated with the described anomalies as well as the anomalous date and machine serial number.      
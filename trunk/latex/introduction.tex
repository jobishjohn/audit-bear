\section{Introduction}

Currently, in the United States, Direct Recording Electronic (DRE) units are used widely. A DRE is a type of electronic voting machine in which the voter interacts directly with the machine, typically through a touch screen. DREs provide a friendly interface to assist the voter with the ballot marking process. Compared to other widely used voting systems such as optical scan systems, DRE units can prevent overvoting and reduce undervoting. Unique to DRE machines, electronic ballots can be issued on demand; due to this feature, poll workers will never run out of ballots at the precinct.  Additionally, audio DREs can assist visually impaired voters.
 
Federal standards require that electronic voting machines generate detailed logs, which can be used during post-election audits. These audit logs record events as they occur on the voting machine, such as opening the machine for voting, casting a vote or closing it at the end of election day. The log data may also include an electronic record of every ballot cast in the voting machine.  Previous work has have shown how these logs can be analyzed to uncover procedural errors and anomalies that occur during the election [Buell].  Unfortunately, manual analysis of raw log data is usually cumbersome and time consuming, making county-wide post-election analysis impractical and prone to human error. Therefore, at the present time, election officials do not regularly perform these types of analyses. 

We aim to make DRE audit log analyses more useful and more accessible to election officials and other interested parties. In this work, we developed new methods for analysis of these audit logs. Our work demonstrates new uses of audit log data and extends the range of election problems that can be detected this way to enable others to benefit from this research; we created a web application, freely available to the public, that applies our methods to detect procedural errors and system deficiencies.  Election officials can use our tool to identify memory cartridges containing precinct totals that were not uploaded on election night, machines that may have experienced hardware problems during the election, polling locations that closed late or had voters waiting in line for extended periods.
 
Our research builds on a similar study conducted with DRE audit data collected by fourteen South Carolina counties during the 2010 primary and general elections.  The authors of that study were able to determine, solely by analyzing the audit logs, that 1127 votes did not get included in the official certified tally in Richland County, South Carolina [Buell]. These findings were possible because DRE systems used in South Carolina produce three different types of audit logs, each capturing slightly different information. By cross checking the logs against each other, the authors found inconsistencies that enabled them to uncover the missing votes. In our research we used the same data set and were able to replicate their results. We took this matter further and found fifteen memory devices containing votes that were not uploaded to the tabulation systems from seven counties during the 2010 General election. These memory devices tallied 2082 total votes. Without additional information we could not verify whether alternate procedures were used to add these missing votes to the aggregated totals. 

Our methods for analysis of audit logs are applicable generally to all DRE voting systems  that produce the necessary audit logs. We implement these ideas for ES\&S iVotronic  DREs and focus especially in them as an example target for analysis.  We chose to work with iVotronic audit logs because they were readily available to us. The iVotronic system is a standalone, portable, touchscreen system that records vote totals, ballot images and an event log on internal flash memory. The event log records, in chronological order, the system events including unit configuration, polls opened, votes cast, polls closed, calibration or battery issues, warnings and system errors. iVotronic voting machines are used throughout the US, in 422 jurisdictions tallying more than 22 million registered voters [cite exact page of verifiedvoting.org], and represent one of the most widely deployed DREs in the US.
 
A brief description of several problems we detect follows.

\textbf{Votes not uploaded.} We detect  memory cartridges used to close voting machines that have not had their vote data uploaded to the tabulation system. This situation, if not corrected, can result in votes that were never included in the official results and thus not counted. Our tool produces a report containing the precinct's name, the serial number of the iVotronic voting machines whose votes were collected in the cartridge, and the total number votes stored on the cartridge. With this information, the election officials can quickly correct the problem by locating and uploading the missing cartridges into the cumulative totals.

\textbf{Machines not closed.} We detect voting machines that were not closed at the polling location. Failure to close a machine on election night may result in its votes not included in the certified count. Our analysis outputs the precinct's name and voting machine's serial number. With this information, election officials can quickly locate the voting machines, close them out, and then upload their votes to the cumulative totals.

\textbf{Missing terminals from the audit database.} This analysis identifies any iVotronic voting machines used during the election whose event log or ballot images have not been uploaded to the election reporting software. Our tool provides election officials the information needed to locate the voting machines or removable media containing the missing audit data so that the files can be uploaded to the election reporting software. Complete DRE ballot images and event logs will allow for more accurate and complete post-election audits.

\textbf{Polling location related analyses.} Our tool provides a series of analyses related to polling location activity. We identify locations that stayed open late as well as locations that may have experienced long lines during the day. This information can help county officials to identify locations that may need additional resources in the future. 

\textbf{DRE voting machine configuration and hardware problems.} Our tool performs several analyses that can identify  voting machines that may need testing, repair or reconfiguration. These analyses include identifying possible calibration issues, machines with potential power supply issues, machines that were forced to close early, and machines with incorrect date and time settings.

\textbf{Poll worker training related issues.} We also identify incorrect procedures at the precincts such as using the wrong cartridge to close the voting machines in a precinct, forgetting to print the precinct's zero tape or activating ballots with the incorrect cartridge. Election officials may be able to use this information to improve pollworker training and minimize recurrences in the future.

We hope our methods will help election officials to improve poll worker training, plan for future elections by augmenting resources where they may be lacking, and schedule systems upgrades or maintenance based on failures identified through machine audit logs, thus enhancing the electoral process and voter confidence.

In this work we assume that DRE audit logs are complete, accurate,  trustworthy and free of accidental or malicious tampering. Detecting and preventing audit log tampering is outside of the scope of this work.

In summary, this paper develops and implements new ways that audit log data can be used meaningfully and in an automated fashion to enhance the accuracy and efficiency of elections. We believe our tool will provide useful feedback to election administrators during the canvassing process. We hope that this study illustrates the potential value of voting systems' audit logs and motivates future election technologies to provide enhanced support for these purposes.


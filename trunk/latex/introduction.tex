\section{Introduction}

Currently, in the United States, Direct Recording Electronic (DRE) units are used widely. A DRE is a type of electronic voting machine in which the voter interacts directly with the machine, typically through a touch screen. DREs provide a friendly interface to assist the voter with the ballot marking process. Compared to other widely used voting systems such as optical scan systems, DRE units  can prevent overvoting and reduce undervoting. Additionally, audio DREs can assist visually impaired voters.
 
However, one drawback of using a paperless DRE system is that it does not generate a paper trail for verification of  voter's intent. As a result, Federal standards require that electronic voting machines generate detailed logs which can be used during post-election audits. These logs record events as they occur on the voting machine, such as �terminal opened� , �vote cast� and �terminal closed�. The log data may also include an electronic record of every ballot cast in the terminal.  The authors of the paper �Auditing a DRE-Based Election in South Carolina� [1], have shown how these logs can be analyzed to uncover procedural errors and anomalies that occur during the election.  Manual analysis of raw data is usually cumbersome and time consuming, making countywide post-election analysis impractical and prone to human error. Therefore, at the present time, election officials do not regularly perform these types of analyses. 

We aim to make DRE audit log analyses more useful and more accessible to election officials and other interested parties. In this work, we created a web application, available to anyone, that performs a variety of analyses on election data to detect procedural errors and system deficiencies. Our tool allows election officials to use machine audit logs to recreate election day events, identify possible voting machine errors or system unexpected behavior and to identify areas of election day protocol that may need improvement. 
 
Our research parallels with and builds on a similar study conducted with DRE audit data collected by fourteen South Carolina counties during the 2010 primary and general elections.  The authors of that study were able to determine, solely by analyzing the audit logs, that 1127 votes did not get included in the official certified county in Richland county[1]. These findings were possible because DRE systems used in South Carolina produce three different types of audit logs, each capturing slightly different information. By cross checking the logs against each other, the authors found inconsistencies that led them to uncover the missing votes. In our research we used the same data set and were able to replicate their results. We also found: 

[Brief description of one of our findings.]

Our research was conducted with data from the iVotronic DRE system which is manufactured by Election Systems and Software (ES\&S).  The iVotronic system is a standalone, portable, touchscreen system that records vote totas, ballot images and an event log on internal flash memory. The event log records, in chronological order, the system events including unit configuration, polls opened, cast votes, polls closed, calibration or battery issues, warnings and system errors. 
 
This study proposes analyzing DRE audit data in an automated fashion to enhance the accuracy and efficiency of post- election audits. Election officials can use the analyses our toll reports to identify memory cartridges containing precinct totals that were not uploaded on election night, machines that may have experienced hardware problems during the election,, locate polling locations that experienced lines of voters and determine which locations closed late. A brief description of our analyses follows.

\textbf{Votes not uploaded.} This analysis warns election officials of any memory cartridge(s) used to close iVotronic terminals that not had their data uploaded to the tabulation system. Our tool produces a report containing the precinct's name, the serial number of the iVotronic terminal(s) collected in the cartridge and the total number votes stored on the cartridge. With this information, the election officials can quickly locate and upload the missing cartridge(s) into the cumulative totals resulting in accurate election night reporting.

\textbf{Machines not closed.} This analysis produces a report listing the terminals that were not closed at the polling location. This analysis outputs the precinct's name and iVotronic(s)' serial number. With this information, election officials can quickly locate the terminals, close them out, and then upload their votes to the cumulative totals.

\textbf{Missing terminals from the audit database.} This analysis identifies any iVotronic terminal(s) used during the election whose event log or ballot images have not been uploaded to the election reporting software. Election officials can locate the terminal(s) or removable media containing the missing audit data so that the files can be uploaded to the  database of the election reporting software. Complete DRE ballot images and event logs will allow for more accurate and complete post-election audits.

\textbf{Polling location related analyses.} Our tool provides a series of analyses related to polling location activity. We identify locations that closed late as well as locations that may have experienced long lines during the day. This information can help county officials to identify those locations which may need additional resources in the future. 

\textbf{DRE terminal configuration and hardware problems.} Our tool performs several analyses that can identify iVotronic terminals that may need testing, repair or reconfiguration. These analyses include identifying possible calibration issues, machines with low batteries, terminals that were forced to close early and machines with incorrect date and time settings.

\textbf{Pollworker training related issues.} Four analyses are available on our website which can be used to enhance the pollworker training curricula. These analyses identify incorrect procedures at the precincts such as: using the wrong cartridges to close terminals in a precinct, failure to print the precinct's zero tape, activating ballots with the incorrect cartridge(s) and statistics of different reasons for canceling ballots. 

Auditing DRE log files can be conducive to accurate certified election results. Expediting post-election analyses become particularly critical in instances where the time between elections is short, as is typical in run-off elections. The availability of data compiled in a systematic way permits election officials improve the pollworker training, plan for future  elections by augmenting resources where they may be lacking and schedule systems upgrades or maintenance based on failures identified through machine audit logs, all enhancing the electoral process and voter confidence.

We based our study solely on the audit data logs which are considered the ground truth for election analyses and disputes. For this reason, it is important that audit logs are protected from accidental or malicious tampering. However, in this work we assume that DRE audit logs are complete, accurate and trustworthy. Detecting and preventing audit log tampering is a field of research outside of the scope of this study.

In summary, this study implements ways in which this data can be used meaningfully and in an automated fashion to enhance the accuracy and efficiency of elections. We believe our tool will provide intelligent feedback to election administrators during the canvassing process and post-election audits. We hope that this study serves to influence similar audits that can be expanded to other election technologies.


\subsection{Procedural Errors}
Our tool can detect procedural errors and poll worker mistakes.  A few of these are: precincts that do not print zero tapes on the morning of election day; using a master PEB to activate ballots; opening and closing machines with different PEBs.  

\subsubsection{Printing Zero Tapes}
According to the South Carolina poll worker training video (citation), poll workers are required to print at least one zero tape per polling location on the morning of the election.  Using the event log, our tool checks each polling location for this event and reports the locations that did not record this event.  

\subsubsection{Ballot Activation with a Master PEB}
Another way our tool finds procedural errors is by crosschecking the master PEBs with the PEBs used to activate ballots.  Poll workers should be using non-master PEBs to activate ballots so that the PEBs do not get switched.  

\subsubsection{Opening and Closing a Machine with Different PEBs}
Along the same lines, we report incidents of opening and closing a machine with different PEBs.  A machine should be opened and closed with the same master PEB; if not, it may be more likely that this PEB does not get uploaded.   

\subsubsection{Anomalous Vote Cancelled Events}
When poll workers cancel ballots, they must select a reason why; this is another way to detect errors.  There are seven options for canceling a ballot: wrong ballot, voter left after the ballot was issued, voter left before the ballot was issued, voter request, printer problem, terminal problem, or an unspecified reason.  If there are any instances of canceling a ballot due to a printer problem, it could be an indicator of a procedural error because ballots are not printed.  In other cases, if there is a large number of a specific reason, such as having the wrong ballot, this could indicate the poll workers are repeatedly issuing the wrong ballot.  




\subsubsection{needs to be split up for the previous sections}
It may be beneficial to election officials if they could detect at which locations poll workers are following the required procedures.  Procedural errors can cause many problems including lost votes, incorrect vote counts, disgruntled voters, and long lines.  If election officials are aware of the procedures that are not being followed, they could review their precinct checklist.  This will allow for more efficient audits as well as a better voting experience for voters.  

While our analyses detect an important set of errors, there are certainly many more procedures that can be analyzed.  In addition to printing zero tapes in the morning, poll workers are required to print results tapes at the end of the election; unfortunately, this is not detectable due to the way the event log is produced.  We have inferred from the event logs that the poll workers are extracting the compact flashes before printing the results tapes, therefore the event log shows no record of the event.  



\subsection{Incomplete Audit Data}
Due to the manner in which both the event log and the ballot image file are created, they may not include all audit data.  Election officials manually create the two databases (one for each log) in ERM after the CF cards containing the audit data are uploaded to the ERM servers.  When a machine occurs in the event log with recorded votes cast on it during election day, but does not appear in the ballot image file, this indicates that some ballots are missing from the ballot image database.  The reverse situation reveals the opposite procedural error- the event log is not complete.  We developed a technique that identifies incomplete databases.  Incomplete DRE ballot images will prevent accurate manual recount procedures.

This analysis requires an event log and a ballot image file, where both files record machine serial numbers for any events and ballots cast.  Our analysis crosschecks these files for inconsistencies that may reveal unforeseen problems.  In our algorithm, we create two data structures; the first contains every machine used for voting and how many votes were cast on it according to the event log, the second contains every machine with ballot images on it and how many ballots were cast on it according to the ballot image file.  Then, we compare these two structures.  Ideally, they would have the same contents and we could conclude that the files are consistent.  However, there may be cases where one file has a machine with more votes on it than the other file�s corresponding machine.  In this case, we would report file inconsistencies and provide the user with the machine serial number and the conflicting vote counts.       

\smsubsection{Votes Possibly Not Counted}
\smsubsubsection{PEBs Not Uploaded}
\label{sec:pebs_not_uploaded}
The precinct procedures for poll workers dictate that a single PEB
should be used to open and close all machines at a polling
location. Failure to strictly follow this protocol can lead to problems if
election workers at county headquarters upload only one PEB, expecting all
machines to be closed with the same PEB. Recall that the PEB contains the
official cumulative tally for the precint. This has happened in the past and it 
causes significant delays in the reporting of election
results~\cite{Buell2011,Mazella2002}. 

Our
tool provides an analysis that generates a list of PEBs used to collect
votes on election day. It warns election officials of any PEB, master
or non-master, that was used to close a terminal, whose data was not
uploaded to the election reporting manager (ERM) system.  

This analysis is applicable to any voting system that has components
equivalent to a PEB and a voting terminal.  It requires an event log
that records the serial number of the PEB used to close every voting
machine and the vote events processed in the voting machine, in
chronological order. It also requires a system log file that records the
serial number of every PEB uploaded to the ERM 
system. Finally, a log providing a list of terminals used in each
precinct is also required.

Our method keeps track of the serial number of the PEB used to close each voting
machine, the polling location each voting machine was assigned
to, and the total votes cast in each machine. Then, it
reports any PEB containing vote data that has not been added to the cumulative
count.  For each such PEB, our analysis reports the serial numbers of the
terminals collected by the PEB, the number of votes processed in those terminals
and the precinct's name and number. With this information, election officials
can gather the missing PEBs and collect votes from terminals not included in the
cumulative totals.

\smsubsubsection{Machines Not Closed}
Votes could also be left out of the official tally if poll workers faill to close
all voting machines. If a machine is not closed, then a PEB has
not collected its data and its votes can not be counted; our algorithm
will  detect this situation.   

This analysis requires an event log with recorded events marking the
opening and closing of each voting machine. It also requires a log
file that allows us to identify which machines were at each polling
location. In our analysis we use the event log and the ballot image
file for this information. We created a method that checks if a
machine was closed, given it was also opened for voting.  
If there are any machines that have been opened and not
closed, they are displayed to the election official.  These
inconsistencies are only detected if the machines are closed when the
CF card is still in the machine and the complete audit data, including
the closure event, is saved to the CF card and the CF card gets uploaded to the
ERM.


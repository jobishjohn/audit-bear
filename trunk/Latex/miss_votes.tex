\subsection{Votes Possibly Not Counted}
\subsubsection{PEBs Not Uploaded}
The precinct procedures for poll workers dictate that a single PEB
should be used to open and close all machines at a polling
location. Failure to strictly follow this protocol led to problems in
Miami-Dade County during the 2002 Primary
election~\cite{Mazella2002}. In that case, poll workers used two or
more PEBs to open and close terminals at their precinct.  However,
election workers at county headquarters only uploaded one of these
PEBs, expecting all machines to be closed with the same PEB. This
caused a significant delay in the reporting of election results. A
recent study found similar problems in several South Carolina
counties~\cite{Buell2011}.

Currently, the election officials performing the vote tabulation on
election night do not have a way to effectively compare the list of
PEBs that have been successfully uploaded to the tabulation software
with the pool of PEBs containing vote data throughout the county. Our
tool provides an analysis that generates a list of PEBs used to collect
votes on election day. It warns election officials of any PEB, master
or non-master, that was used to close a terminal, whose data wast not
uploaded to the election reporting system.  

This analysis is applicable to any voting system that has components
equivalent to a PEB and a voting terminal.  It requires an event log
that records the serial number of the PEB used to close every voting
machine and the vote events processed in the voting machine, in
chronological order. It also requires a system log file that records the
serial number of every PEB uploaded to the ERM tabulation
system. Finally, a log providing a list of terminals used in each
precinct is also required.

Our method keeps track of the serial number of the PEB used to close each voting machine. It also records the polling location each voting machine was assigned to as well as the total votes cast in each machine. Then, it verifies that each PEB that was used to close voting machines is uploaded to the ERM vote tally. It reports any PEB containing vote data that has not been added to the cumulative count.  For each such PEB, our analysis reports the serial numbers of the terminals collected by the PEB, the number of votes processed in those terminals and the precinct's name and number. With this information, election officials can gather the missing PEBs and collect votes from terminals not included in the cumulative totals.

\subsubsection{Machines Not Closed}
There are two main pieces of the election system that need to be
analyzed to determine if votes were left out of the count.  We have
discussed the first essential piece: the PEB.  The second piece of the
election system that must be taken into account is the voting machine
itself.   We have devised a way to determine which machines have not
been closed for voting.  If a machine is not closed, then a PEB has
not collected its data and its votes can not be counted; our algorithm
will  detect this situation.   

This analysis requires an event log with recorded events marking the
opening and closing of each voting machine. It also requires a log
file that allows us to identify which machines were at each polling
location. In our analysis we use the event log and the ballot image
file for this information. We created a method that checks if a
machine was closed, given it was also opened for voting.  Our analysis
searches through the event log for machines that were opened; these
machines are stored in a data structure.  Next, we check that every
machine in the data structure also recorded an event representing its
closure.  If there are any machines that have been opened and not
closed, they are displayed to the election official.  These
inconsistencies are only detected if the machines are closed when the
CF card is still in the machine and the complete audit data, including
the closure event, is saved to the CF card.  If the audit data is not
complete, we cannot determine whether a machine was closed or not.
There may be other cases that are undetectable by our analysis because
a machine's audit data will not be in the event log or ballot image
file if it has not been closed in the normal circumstances or the CF
card was never uploaded to the ERM.  

\subsection{Systematic Date and Time Errors}\label{an:date} Correct time-stamps
are critical in post election audits as they can preclude further log analysis.
We found it simplest to classify these date errors into two categories: errors
resulting from incorrectly set clocks and errors resulting from apparent bugs in
the time-stamp mechanism itself. The algorithm categorizes and generates a list
of these errors for inclusion in a web report. False positives were avoided by
testing the algorithm against the iVotronic data-set and tweaking threshold values.
The algorithm is not specific to these logs but the implementation in this
project is.

This analyses assumes certain properties of the audit logs hold true. Each event
is marked with a time-stamp based on the machine's internal clock. Each clock is
manually set.  Any manual adjustments made to the clock are recorded on the
audit log. Additionally, opening and closing events recorded in the logs proved
useful to the analysis. This allowed the algorithm to related date and time
issues with election day voting.  Otherwise, some of the reported errors might
not be relevant. The iVotronic data used in development was found to
occasionally violate these properties so checks were developed to report these
violations as well.

Emphasis was placed on the accurate identification of date errors rather then
complete identification. False positives create excessive amounts of data which
would undermine the usefulness of the analysis. Determining the accuracy of
time-stamps during a post-election audit can have a lot of ambiguities. If a
machine has its clocks set an hour back it is difficult to determine if it
opened early or if its clock is incorrectly set. For these reasons, techniques
were developed in such as way as to avoid as many false identification of errors
as possible.

All data was collected by sequentially parsing each time-stamp while looking for
specific events such as manual time changes or erroneous shifts between
time-stamps. A machine's open state was tracked by looking at terminal opening
and closing events. Only the last set of openings and closings were considered
for each machine. This helped ensure we were looking at election day statistics.
Early voting was found to be conducted at inconsistant dates among the counties
and was ignored for accuracy and relative importance of events such as manual
time changes.


\subsubsection{Incorrectly Set Dates}
This report identifies any machine determined to open for election day voting
with an incorrect clock.  This means the opening event and possibly all vote
events were recorded with incorrect time-stamps. This can prevent other analyses
from properly detecting how long the polls stayed open, whether they opened on
time, or if they experienced long lines. Additionally, It may be useful for an
election official to identify counties or precincts where these errors were
particularly common so that pre-election testing procedures can be improved.
There are two tables to categorize the results for this report.

The first table identified machines that had their clocks adjusted during
election day voting. This caught machines that were only off by a couple hours
as well as machines opening on election day with completlely incorrect dates.
Machines whose clock is only incorrect by a couple hours are difficult to
identify because opening and closing times are not a consistant point of
reference. Because of this ambiguity, only machines who were manually adjusted
during election day were marked. The algorithm looked for a manual time
adjustment event for any machine that was in an open state during election day.
If the machine then closed successfully on election day, the delta of the time
change is recorded. This time delta can be used to correctly adjust the recorded
opening time of the machine.

The second table found machines that opened on incorrect dates for voting but
never had their clocks corrected. In the first table, clocks that were only
slightly off were detected only because someone changed the date during
elections.  These small changes couldn't be detected if there wasn't an obvious
manual adjustment. To detected machines whose clock was never corrected a
threshold was determined to mark machines opening and closing on incorrect
dates. Dates a month before election day and any dates one or more days after
election day were considered incorrect.  Based on the data set used this
appeared to yeild only machines that indeed went through election day voting
with an incorrect clock.

\subsubsection{Datetime Errors}
The second report identifies anomalies in the time stamps that aren't a result
of human error. We found many dates that changed seemingly at random and without
a manual date adjustment event. These occurrences suggest an issue with the
machine itself rather then procedural error. The cause of these errors remain
unknown and have the potential to invalidate other audit statistics. It is
concerning that these set values can change arbitrarily. Reporting these
anomalies may assist offcials in determining which machines are experiancing
errors to potentially report to the DRE vendor.

Many machines were found to exhibit odd behavior in which a clock would appear
to temporarily or permantly shift its clock without human adjustment. This
algorithm identifies these jumps.  In all the cases observed these jumps in
clock values were significant enough to not be confused as the passing of time
between two events in the log. 

Since events are appended in chronological order; any time-stamp that decreases
chronologically without a manually date adjustment event is wrong and added to
this category. Forward jumps in time are not necessarily the result of an
inconsistent time record because a machine can be shut down for weeks between
events.  Given the relatively short duration of elections, jumps forward in
time, that are past a certain threshold, can reasonably be considered erroneous.
The algorithm looked for deltas between time-stamps that jump forward by more
then 33 days.  Given the date set used, this number seemed to avoid listing
machines simply not used between the factory setup and election day voting.
Additionally, some machines were found to have their clock set to some type of
null state where the clock stopped incrementing and marked blank dates. 

Once one of these odd changes in clock value is detected, the previous
time-stamp is marked as the start and a counter increments on each subsequent
event.  The counter ends and the anomaly is assumed to have ended based on
criteria for each type of jump. The backward jump is presumed over if an event
appears that is chronologically forward in time compared to the start event.
The forward jump is assumed over if an event appears on the same day of the
starting event.  The zero clock anomaly is detected as returning to normal as
soon as a non-zero date appears. The number of events associated with an anomaly
is reported as well as the anomalous date and machine serial number. 


\bigvertspace
\section{Introduction}
\smvertspace
A Direct Recording Electronic (DRE) voting machine is one in which the 
voter interacts directly with the terminal, typically through a touch
screen. DREs provide a friendly interface to assist the voter with the
ballot marking process. Similar to the commonly used optical scan
systems, DRE units can reduce overvoting and undervoting. Uniquely,
DRE machines can issue electronic ballots on demand; running out of
paper ballots is no longer an issue. Additionally, audio DREs can
assist visually impaired voters.
 
Federal standards require that electronic voting machines generate
detailed audit logs for use during post-election
audits. These logs record events as they occur on the voting machine 
such as opening the machine for voting, casting a vote or closing the
machine at the end of election day. The log data may also include a
record of every ballot cast in the voting machine.  Previous work has
shown how these logs can be analyzed to uncover procedural errors and
anomalies that occur during the election\cite{Buell2011}.
Unfortunately, manual analysis of raw log data is usually cumbersome
and time consuming, making county-wide post-election analysis
impractical and prone to human error. Therefore, at the present time,
election officials do not regularly perform these types of analyses. 

We aim to make DRE audit log analysis more useful and accessible to
both election officials and other interested parties. In this work, we
develop new methods to analyze audit logs for the detection of
both procedural errors and system deficiencies. We created a public
web application that provides our analyses as a free service for use by election
officials or interested third parties.
 
Our research builds on a similar study that was conducted with DRE audit data
collected by fourteen South Carolina counties during the 2010 primary and
general elections.  The authors of that study were able to determine, solely by
analyzing the audit logs, that 1127 votes did not get included in the official
certified tally in Richland County, South Carolina~\cite{Buell2011}. In our
research we used the same 
data set as a basis for development of our software. First, we replicated the
detection of votes not uploaded. We took this matter further and found fifteen
memory devices containing votes that were not uploaded to the tabulation systems
from seven counties during the 2010 General election. These memory devices
tallied 2082 total votes. Without additional information we could not verify if
alternate procedures were used to add these missing votes to the aggregated
totals. 

We implement these methods for the ES\&S iVotronic DRE as the 2010 South
Carolina data was already publically available through a previous
Freedom of Information Act request and the iVotronic was used in that election. 
%\cks{Is this true? I changed the wording, but I'm not actually sure   how the
%data was made available.} 
The iVotronic system is a
standalone, portable, touchscreen system that records vote totals,
ballot images and an event log on internal flash 
memory. 
iVotronic voting machines represent one of the most widely deployed
DREs in the U.S. In 2010, 422 jurisdictions tallying more than 22
million registered voters used this system~\cite{VerVot2010}. However, our
general methods for analysis of audit logs 
are applicable to all DRE voting systems  that produce the necessary
audit logs.  

A brief description of several problems we detect follows.
\smvertspace
\begin{itemize}
\item We detect situations that, if not corrected, can result in votes left out
of the official results.  
\item We developed several analyses to identify voting machines that may need
testing, repair or reconfiguration. 
\item We identify instances of incorrect procedures being followed at the
precincts. Election officials may be able to use this information to improve 
poll worker training and minimize likely sources of human error in the future. 
\item  We identify locations that stayed open late, which can help county
officials identify locations that may need additional resources in the future.  
\item We also identify voting machines used during the election, but whose audit log data 
have not been uploaded to the election reporting software, potentially causing
inaccurate post-election audits.  
\end{itemize}
In this work we assume that DRE audit logs are complete, accurate,  trustworthy, and free of accidental or malicious tampering. Detecting and preventing audit log tampering is outside of the scope of this work.

\begin{comment}
In summary, this paper develops and implements new ways that audit log data can
be used meaningfully and in an automated fashion to enhance the accuracy and
efficiency of elections. We believe our tool will provide useful feedback to
election administrators during the canvassing process. We hope that this study
illustrates the potential value of voting systems' audit logs and motivates
future election technologies to provide enhanced support for these purposes.
\end{comment}




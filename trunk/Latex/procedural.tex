\subsection{Procedural Errors}
\smvertspace
It may be beneficial to election officials if they could detect at which
locations poll workers are following the required procedures. Procedural errors
can cause many problems including lost votes, incorrect vote counts, disgruntled
voters, and long lines. A few errors that might occur are: precincts that do not
print zero tapes on the morning of election day; using a master PEB to activate
ballots; and opening and closing machines with different PEBs. If election
officials are aware of the procedures that are not being followed, they could
improve their precinct checklist. This will allow for more efficient audits as
well as a better voting experience for voters. 

\smsubsubsection{Printing Zero Tapes}
This analysis pertains to jurisdictions that require poll workers to
print at least one zero tape per polling location.  According to the
South Carolina poll worker training video~\cite{VotingInfo}, poll
workers are required to print at least one zero tape per polling
location on the morning of the election. The printing of a zero tape
is important to verify that the voting machines are set to zero; it
ensures that additional votes will not be included in the certified
count. In this analysis we check if this procedure is followed.

In order to run this analysis, we require both a ballot image file, which contains the machines and precincts, and an event log.  Given an event log that has correct date and time stamps, and an event that represents the printing of a zero tape, we can detect if this procedure was followed.  We limit our search of the event log to events that occurred on election day.  We have noticed that the zero tape may be printed before or after the official poll opening time, so we do not specify time limitations.  Our analysis groups the machines found in the event log by polling location using the ballot images file.  Then we crosscheck the list of machines with the specified event; our analysis requires at least one machine per polling location to have this event.  Poll workers need to print the zero tape after they have opened all terminals with the same PEB; they must use this same PEB to open the last machine and print the tape.  If none of the machines in a polling location record this event, we report the polling location and the corresponding machines.

\smsubsubsection{Ballot Activation with a Master PEB}
Another procedure that should be avoided is using a master PEB to activate ballots for voters.  Poll workers should be using non-master (activator) PEBs to activate ballots for the reasons outlined in Section \ref{sec:background}.

This analysis focuses on procedures related to the use of the PEB and the voting machine; it is applicable to other voting systems that use a similar process.  To detect violations of this procedure, the analysis requires that the event log records votes being cast, machines being opened and closed, as well as the PEB number used for each event.  Through this analysis we assign each machine a master PEB, which is the PEB poll workers are trained to open and close the machine with; by doing this, we are assuming that poll workers are following correct procedures.  Then, we check to see if any votes were activated on each machine using the master PEB.  If we find any such cases, we report them.  In the case that poll workers did not follow procedures and a different PEB was used to close the machine than was used to open the machine, our algorithm checks both the opening and closing PEB.  If a ballot was activated with either the closing or opening PEB, then that machine and PEB are reported.  

\smsubsubsection{Opening and Closing a Machine with Different PEBs}
\label{sec:opening_closing_diff_pebs}
Along the same lines, we report incidents of opening and closing a
machine with different PEBs. A machine should be opened and closed
with the master PEB; if not, it may be more likely that the non-master
PEB used for closing machines and its corresponding votes will be left behind at the precinct.  Poll workers are trained so that they put the master PEB, CF cards, and printed precinct tape in a designated bag, which is transported to election central after the precinct is closed.  

The requirements for this analysis are similar to the analysis discussed above.  As this analysis works with procedural errors concerning the interaction between PEBs and the machine, it applies to systems that work in the same manner.  The event log should contain events signaling the opening and closing of each machine and the PEB used for each event.  In this analysis, we find the PEB used to open a machine and the serial number of the PEB used to close that machine.  If the two PEBs have conflicting numbers, then we report the machine and PEB serial numbers.  If the PEBs are the same, then the correct procedure was followed and no machines are reported.  

\smsubsubsection{Anomalous Vote Cancelled Events}
When poll workers cancel ballots, the voting machine displays a menu with several ballot cancellation explanations and poll workers must select one of these reasons; this is another way to detect procedural errors. There could be many reasons why votes are cancelled. According to the iVotronic’s event logs there are seven options for canceling a ballot: \textquotedblleft wrong ballot,\textquotedblright \hspace{1 mm} \textquotedblleft voter left after the ballot,\textquotedblright \hspace{1 mm}  \textquotedblleft voter left before the ballot,\textquotedblright \hspace{1 mm}  \textquotedblleft voter request,\textquotedblright \hspace{1 mm}  \textquotedblleft printer problem,\textquotedblright \hspace{1 mm}  \textquotedblleft terminal problem,\textquotedblright \hspace{1 mm}  or \textquotedblleft other reason.\textquotedblright \hspace{2 mm}  Depending on the reason chosen, it might be possible to identify procedural errors.  For example, if there are any instances of canceling a ballot due to a printer problem, it could be an indicator of a procedural error because ballots are not printed unless the voting machine is connected to a voter verifiable print out module. In other cases, if there is a large number of a specific reason, such as having the wrong ballot, this could indicate the poll workers are repeatedly issuing the wrong ballot or the terminal is experiencing calibration issues.  

This analysis has a higher potential to be used on other voting systems than the previous two analyses.  This analysis only requires that the event log record votes being cancelled, and reasons as to why the vote is being cancelled.  This analysis would perform best if the date and time stamp were accurate as well.  For each machine, we associate it with how many instances of each type of vote cancellation are recorded.  We expect some \textquotedblleft normal\textquotedblright \hspace{1 mm} number of instances of each vote cancellation event, but we do not know what this \textquotedblleft normal\textquotedblright \hspace{1 mm} number is.  To find this we need the average number of each vote cancellation type; therefore, the analysis has seven averages to calculate.  Next we calculate the standard deviation for each of these event types.  After analyzing each machine's event history, we report the machines that had at least one vote cancellation event type that occurred an unusually high number of times. In our analysis, an \textquotedblleft unusually high number of times\textquotedblright \hspace{1 mm} is defined as having a number of instances greater than three standard deviations above the average.  If it is the case, we will also report if a machine records this unusually high number of instances for more than one vote cancellation event type, and which event types they are.  


\subsection{Future Analysis}
Election officials assign voting machines and supplies to each polling
location based on the number of voters registered in the precinct.
However, voter turnout can vary and as a result, some polling
locations may end up overstocked with equipment, supplies or poll
workers while others may lack resources on election day. As a result,
voters may have to wait in line before casting their vote. This is
common during mid-term and general elections~\cite{Kreitman2010,
  Slade2008, U2010}, as well as any elections with high-voter turnout.
In those circumstances, election officials might like information
about which locations experienced long lines and at what time of the
day. However, monitoring all the polling places in a large county can
be a daunting task. Therefore, providing an analysis that can analyze
DRE audit data to identify peak times at the precincts would be useful; such
information could assist election officials with the planning of
resource allocation for future elections.

\begin{comment}
This analysis reports busy locations by detecting heavily used voting terminals. It requires an event log that captures every vote cast event along with an accurate time stamp and precinct location information. In our analysis we find this information in the event log and the ballot image file. When there are consecutive ballots cast in all machines in a polling location with no time delay in between, we are able to infer that there is a steady flow of voters and possibly a line of voters waiting to use the voting machines. 
\end{comment}

However, the analysis is not quite so straightforward. The event logs
we used captured the \textquotedblleft cast vote'' event, but that
only tells us when the voter finished making their selection. In order
to know if there is any idle time between voters, we would also need
an event that notes when a new ballot is loaded into a machine. The
idle time would be the time between one \textquotedblleft cast vote''
event and the next \textquotedblleft ballot loaded''
event. Unfortunately our logs have no such \textquotedblleft ballot
loaded'' event. We are exploring the possibility of inferring the idle time by
comparing the time between cast vote events in any given time period to the time
between cast vote events at times when we can reasonably assume voters were
waiting in line and the idle time was likely close to zero.
\begin{comment}
Instead, we had to infer the idle time. We did this by
focusing on the polling locations which stayed opened after poll
closing time as we could conclude they were busy processing the voters
standing in line at that time and the time between cast vote events
includes no idle time. For those locations, we measure the time it
took voters to cast ballots during the extended poll hours. We also
calculate and keep track of the time per ballot cast during regular
poll hours. Using the Kolmogorov- Smirnov statistical test we can
determine whether the distribution of time between vote cast events
during regular poll hours, in one-hour time windows, matches the
distribution of time between vote cast events during the extended poll
hours. If the distribution of the two samples is consistent, we can
infer the possibility of long lines during the particular one-hour
time period. \footnote{The KS test starts with the null hypothesis
  that the two samples come from the same distribution, which is the
  situation we are interested in identifying. Therefore, the most we
  can say in the case that the null hypothesis cannot be rejected is, we
  cannot reject the possibility that there were long lines between a
  particular time window.} 
\end{comment}
